\documentclass[a4paper,11pt]{article}
\usepackage[utf8]{inputenc}
\usepackage[T1]{fontenc}
\usepackage[french]{babel}
\usepackage{makeidx}
\usepackage{textcomp}
\usepackage{graphicx}
\usepackage{mathtools,amssymb,amsthm}
\usepackage{lmodern}
\usepackage{multirow}
\usepackage{listings}
\usepackage{array}
\usepackage{longtable}

\title{TER 2019 - Rapport}
\author{Maxime Gonthier - Benjamin Guillot - Laureline Martin}

\begin{document}
\pagenumbering{gobble}\clearpage
\maketitle

\newpage
\tableofcontents

\newpage
\section{Introduction}
	Le projet décrit dans ce rapport est la réalisation d'Algorithmes d'optimisation pour un bureau des temps.\\
	La mission d'un tel bureau est de permettre une décongestion des moyens et infrastructure de mobilité afin d'améliorée la qualité de vie des utilisateur et de diminuer l'impact environnemental de ces structures.
	Il s'agira donc d'agir sur les causes de la congestion de mobilités en repensant les horaires d'activités.
	Dans nôtre projet, les horaires d'activités sont celles de l'université Versailles-Saint Quentin, et plus précisément de l'UFR des Sciences, basé à Versailles. Ainsi en repensant ces horaires, on pourra minimiser la congestion sur la ligne de bus R, allant de Versailles-Chantier à l'UFR.
	Ce rapport a pour objectif de décrire dans un premier temps les données que nous allons utiliser dans notre projet, nous détaillerons les étapes à suivre pour exploiter ces données. Dans un second temps les stratégies de résolution envisageables.\\
	Afin de facilité la compréhension de ce rapport nous allons expliciter les étapes sous la forme d'un exemple simple.
\section{Les données initiales}
\section{Stratégies de résolution}
\section{Conclusion}
\end{document}
