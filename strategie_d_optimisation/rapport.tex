\documentclass[a4paper,11pt]{article}
\usepackage[utf8]{inputenc}
\usepackage[T1]{fontenc}
\usepackage[french]{babel}
\usepackage{makeidx}
\usepackage{textcomp}
\usepackage{graphicx}
\usepackage{mathtools,amssymb,amsthm}
\usepackage{lmodern}
\usepackage{multirow}
\usepackage{listings}
\usepackage{array}
\usepackage{longtable}

\title{TER 2019 - Rapport}
\author{Maxime Gonthier - Benjamin Guillot - Laureline Martin}

\begin{document}
\pagenumbering{gobble}\clearpage
\maketitle

\newpage
\tableofcontents

\newpage
\section{Introduction}
	Ce rapport a pour objectif de décrire dans un premier temps les données que nous allons utiliser dans notre projet, nous détaillerons les étapes à suivre pour exploiter ces données. Dans un second temps nous expliquerons le calcul de la congestion. Enfin nous expliquerons la stratégie d'optimisations que nous utiliserons. Tous cela sera expliciter sous la forme d'un exemple simple.

\section{Les données initiales}
	\subsection{La courbe de trafic}
		La courbe de trafic représente le nombre de personnes qui montent et descendent du bus à chaque arrêt. Dans notre cas d'étude, nous considérons que tous les étudiants montent au premier arrêt (gare des chantiers) et descendent au terminus (l'université). La courbe de trafic est aussi composée des montées et des descentes à chaque arrêt de personnes non-étudiants. \\
		Dans notre exemple nous avons : \\
		\begin{tabular}{ | c | c | c | c | c | c |}
 			\hline			
   			Numéro de l'arrêt	& 1 	& 2 	& 3 	& 4 	& 5\\
   			montées				& 50 	& 15 	& 10 	& 2 	& 0\\
   			Descentes			& 0 	& 5 	& 10 	& 8 	& 54\\
 			\hline  
 		\end{tabular}\\
 		La valeur de montée des non-étudiants au premier arrêt est composée .
 		Pour obtenir les données des arrêts intermédiaires nous choisirons une valeur aléatoire, choisis dans un intervalle différent en fonction de l'heure, l'objectif est de représenter la congestion forte des heures de pointes de manière un peu plus précise : \\
 		\begin{tabular}{ | c | c | c | c | c | c | c | c | c | c |}
 			\hline			
   			Horaire & 7-8h & 8-9h & 9-10h & 10-11h & 11-12h & 12-13h & 13-14h & 14-15h & 15-16h\\
   			Montées & [5:15] & [5:15] & [3:10] & [2:8] & [1:5] & [1:5] & [1:5] & [2:8] & [3:10]\\
   			Descentes & [0:5] & [0:5] & [1:6] & [1:6] & [1:5] & [1:5] & [0:5] & [0:5] & [1:5]\\
 			\hline  
 		\end{tabular}\\

	\subsection{L'emploi du temps}
		L'emploi du temps est un graphe dont les sommets sont des cours et les liens des contraintes entre les cours.
		Pour plus de clarté nous considérons ici que chaque professeur est disponible sur toute la durée de la journée.
		Dans notre exemple nous nous limitons à 10 sommets.
		\begin{enumerate}
			\item On créer un graphe non orienté en numérotant les sommets de 1 à 10.
			\centerline{\includegraphics[scale=0.8]{Captures/exemple1.png}}
			\item On relie les sommets entre eux lorsque qu'il y a une contrainte (même étudiant sur des horaires qui se chevauchent). Pour cela on créé une matrice d'adjacence avec un degré moyen choisis arbitrairement. Pour N sommets et K le degré moyen, la probabilité qu'ily ai une arête à insérer dans chaque case est K/N.
			\item On oriente les arêtes désormais des arcs du plus faible au plus fort
			\centerline{\includegraphics[scale=0.8]{Captures/exemple2.png}}
			\item On remarque dans notre exemple qu'il y a deux sommets de degré entrant nul, ces deux sommets de notre DAG seront des points de départs possible pour notre planification.
			\item On colorie le graphe, chaque couleur représente une salle.\\
			\centerline{\includegraphics[scale=0.8]{Captures/exemple3.png}}
			\item Nous agencons l'emploi du temps en respectant le fait que deux couleurs et deux sommets reliés par un arc ne peuvent pas être sur la même plage horaire. Par défaut le sommet duquel nous partirons est le sommet de degré entrant nul le plus faible, ici le numéro 1.
			\centerline{\includegraphics[scale=0.8]{Captures/exemple4.png}}
		\end{enumerate}
		Chaque cours possède un nombre d'étudiants choisis aléatoirement entre 16 et 32 pour un TD et entre 16 et 100 pour un cours magistral. les TDs représentent trois quart des cours. Ainsi dans notre exemple nous avons : \\
		\\
		\begin{tabular}{ | c | c | c | c | c | c | c | c | c | c | c |}
 			\hline			
   			Numéro du sommet & 1 & 2 & 3 & 4 & 5 & 6 & 7 & 8 & 9 & 10\\
   			Type de cour & TD & CM & CM & TD & TD & TD & TD & TD & TD & CM \\
   			Nombre d'élèves & 16 & 70 & 60 & 30 & 25 & 20 & 16 & 30 & 14 & 50\\
 			\hline  
 		\end{tabular}\\

\section{Calcul de congestion}
	A l'aide du tableau ci dessus nous allons pouvoir déterminé à chaque horaire le nombre d'étudiants arrivé au terminus et donc en conclure à l'aide du tableau des montées et des descentes le nombre de personnes à chaque arrêt. 
	On considère que les etudiants prennent le bus arrivant 15 min avant leurs premier cour de la journée.
	Dans notre exemple nous avons pour  8h le cours 1 et 4, ce qui représente 46 élèves. Ainsi le bus de arrivant a 7h45 ressemble à cela :  \\
	\begin{tabular}{ | c | c | c | c | c | c | c |}
 			\hline			
   			Numéro de l'arrêt & 1 & 2 & 3 & 4 & 5\\
   			Différence montées/descentes & 10 & 7 & 10 & 11 & 0\\
   			Nombre de personnes dans le bus & 56 & 53 & 60 & 56 & 57\\
 			\hline  
 	\end{tabular}\\
 	Ce tableau sera générés pour chaque bus. On mesurera dans chaque cas le dépassement à chaque arrêt en mettant un 1 si il y a dépassements, un 0 sinon.
 	Sachant que la limite de confort est de 50 dans le bus que nous étudions, ce bus nous donne une valeur de congestion de 5.
 	Il faudra pour les bus suivants prendre en compte les élèves déjà présent à l'université avant de calculer le nombre de personnes dans le bus.

\section{Stratégie d'optimisation}
	\subsection{Solution initiale}
		Dans un premier temps nous allons rechercher une solution initiale respectant les contraintes. De la même manière que dans la partie 2.2.6 (plus haut).
		Cette solution va simplement mettre le sommet 1 à 8h puis placer les autres sommets en respectant les contraintes et en les mettant le plus tôt possible.
	\subsection{Mise en rapport de la congestion et des sommets}
		Une approche possible pour trouver une meilleure solution serait d'utiliser une approche gloutonne.
		Il faut dans un premier temps déterminer quels sommets sont les plus impactant en terme de congestion. Une manière de les trier peut être par ordre décroissant de leurs nombre d'étudiants. Ensuite on rechercherais une solution gloutonne où les sommets ayant le plus d'étudiants ont lieu à des horaires où les autres cours ont peu d'étudiants, c'est à dire que nous allons minimiser la somme des étudiants ayant cours au même moments. Evidemment le point de départ sera toujours un sommet de degré entrant nul. Dans notre exemple on aurait pour le tri : \\
		\\
		\begin{tabular}{ | c | c | c | c | c | c | c | c | c | c | c |}
 			\hline			
   			Numéro du sommet & 2 & 3 & 10 & 4 & 8 & 5 & 6 & 7 & 1 & 9\\
   			Nombre d'élèves & 70 & 60 & 50 & 30 & 30 & 25 & 20 & 16 & 16 & 14\\
 			\hline  
 		\end{tabular}\\	
 		\\
 		On place le sommet 1 à 8h car il est de degré entrant nul. Puis on place le sommet 2 à 9h. Ainsi de suite. Une fois arrivé à l'horaire limite de notre exemple (14h) on place les sommets aux endroits où la somme des étudiants sera le plus faible possible. Ainsi le sommet 6 va à la même horaire que le sommet 5, le 7 à la même horaire que le 8, impossible ils ont la même couleur, donc le 7 va a la même horaire que le suivant, le 4 etc...
 		Après l'optimisation gloutonne on aura : \\
 		\centerline{\includegraphics[scale=0.8]{Captures/exemple5.png}}\\
 		On recalcule ensuite la congestion de chaque bus comme vue précedemment.

 	\subsection{Recherche tabou}
 		A partir de la solution initiale nous allons explorer le voisinage de cette dernière pour tenter de trouver des optimums locaux.\\
 		
 		Pour essayer de trouver une meilleur planification et réduire au mieux la congestion on va procéder de la façon suivante :\\
 		\begin{lstlisting}
Pour chaque sommet:
  modifier l'heure tout en respectant les contraintes.
  Pour chaque nouvel horaire:
    calculer la nouvelle congestion.
    Si la congestion est meilleur qu'avant le changement,
      on fixe ce sommet a ce nouvel horaire et on reitere 
      le processus sur les prochains sommets.
    Sinon, on replace le sommet a son horaire initial 
      et on passe au sommet suivant.
		\end{lstlisting}
		Pour chaque sommet, on regarde tout les horaires possibles, même si on trouve une meilleur congestion pendant la recherche. On cherche la meilleur congestion possible pour ce sommet.
		On pourra également utiliser le même procédé en modifiant les sommets 2 à 2 pour essayer de trouver un meilleur résultat.
			
\end{document}
